\section{Introduction}
\label{sec:intro}

Advent of IoT-enabled applications have stimulated the interest for neural networks on edge computing devices.
These are low-cost devices with limited computing and memory capabilities and require low energy footprint.
It has been shown many a times that FPGAs can outperform software implementation of deep neural network~(DNN) as their architecture is friendly to the concurrent architecture of neural networks~\cite{Huimin2016}.
Although FPGA-based DNNs have found their way to datacenters, they are still not popular for edge computing due to their power consumption and limited flexibility.
Hybrid FPGA platforms such as Xilinx Zynq devices are promising in this regard as they are relatively cheaper, low power consuming and can support a hardware-software co-design approach for better flexibility.